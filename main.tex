\documentclass{beamer}
\usepackage{beamerthemesplit}
\usepackage{booktabs}
\usepackage{graphicx}
\usepackage{transparent}
\usepackage{bbold}
\usepackage[italian]{babel}
\usepackage[utf8x]{inputenc}
\usepackage{listings}
\usepackage{tikz}
\usepackage{amsmath,amsfonts,amssymb}
\usepackage{pgfplots}
\usepackage{scalefnt}
\usepackage{color}
\usepackage{xcolor}
\title[PAGERANK]{PageRank}
\institute{
\begin{small}
Corso di Laurea in Informatica Magistrale
\end{small}}
\author{\textbf{Simone Rutigliano}}
\date{\tiny{\today}}

\usebackgroundtemplate{
%    \transparent{0.12}{
     \includegraphics[width=\paperwidth, height=\paperheight]{./figure/escher_hands_tr.png}
%    }
}

%\usetheme{Hannover}
\usetheme{Copenhagen}
\usecolortheme{seahorse}
\usecolortheme{rose}
%\usetheme{Frankfurt}
%\usecolortheme{beetle}

%\useoutertheme[subsection=false]{smoothbars}
%\useoutertheme[subsection=false]{smoothtree}
\useoutertheme{shadow}
\setbeamercovered{dynamic}

\pgfdeclareimage[height=1cm]{logo}{figure/logo}
\logo{\pgfuseimage{logo}}

\begin{document}

%%%%%%%%%%%%%%%%%%%%%%%%%%%%%%%%%%%%%%%%%%%%%%%%%%%%%

\begin{frame}
\maketitle
\end{frame}

%%%%%%%%%%%%%%%%%%%%%%%%%%%%%%%%%%%%%%%%%%%%%%%%%%%%%

\begin{frame}
\frametitle{Outline}
	\tableofcontents
\end{frame}

%%%%%%%%%%%%%%%%%%%%%%%%%%%%%%%%%%%%%%%%%%%%%%%%%%%%%

\section{Introduzione}
\begin{frame}
\frametitle{Definizione}
In 1998 (the same year that HITS was born), Google founders Larry Page and Sergey Brin formulated
their PageRank concept and made it the basis for their search engine [15]. As stated on the Google web
site
By avoiding the inherent weaknesses of HITS, PageRank has been responsible for elevating Google to the
position of the world's preeminent search engine.
\end{frame}

%%%%%%%%%%%%%%%%%%%%%%%%%%%%%%%%%%%%%%%%%%%%%%%%%%%%%

\begin{frame}
	\frametitle{Definizione}
	\begin{itemize}
		\item Formulato nel 1998 da Larry Page e Sergey Brin
		\item Algoritmo di ricerca di Google : " The heart of our software is PageRank \texttrademark\dots it provides the basis for all of our web search tools.''
		\item Supera gli svantaggi di HITS
	\end{itemize}
\end{frame}

%%%%%%%%%%%%%%%%%%%%%%%%%%%%%%%%%%%%%%%%%%%%%%%%%%%%%

\begin{frame}
	\frametitle{Idea di base}
	I voti (link) da siti importanti dovrebbero avere un peso maggiore dei voti (link) da siti meno importanti, e l'importanza di un voto (link) da una qualunque sorgente dovrebbe essere attenuato dal numero dei siti che la sorgente vota
\end{frame}

%%%%%%%%%%%%%%%%%%%%%%%%%%%%%%%%%%%%%%%%%%%%%%%%%%%%%
\section{Funzionamento}
\begin{frame}
	\frametitle{Formalismo}
	\begin{itemize}
		\item Indicata con $P$ una generica pagina
		\item $B_P$ = \{ insieme delle pagine che puntano a P \}
		\item si definisce il punteggio $r(P)$ di $P$: $$\sum_{Q \in B_P}\frac{r(Q)}{|Q|}$$ dove \\ $B_P=\{$ insieme di tutte le pagine puntanti a $P\}$ \\ $|Q|$ = numero degli outlink di $Q$
	\end{itemize}
\end{frame}

%%%%%%%%%%%%%%%%%%%%%%%%%%%%%%%%%%%%%%%%%%%%%%%%%%%%%

\begin{frame}
	\begin{itemize}
		\item Se abbiamo \emph{n} pagine $P_1,P_2,\dots,P_n$ ed assegniamo a ciascuna pagina un arbitrario punteggio iniziale $r_0(P_i)=\frac{1}{n}$
		\item Il punteggio $r(P)$ può essere calcolato mediante la seguente iterazione: $$r_j(P_i)= \sum_{Q \in B_{P_i}}\frac{r_{j-1}(Q)}{|Q|} ~~~~~~~~ j=1,2,3,\dots$$
	\end{itemize}
\end{frame}


%%%%%%%%%%%%%%%%%%%%%%%%%%%%%%%%%%%%%%%%%%%%%%%%%%%%%

\begin{frame}
	\begin{itemize}
		\item Ponendo: $\pi_j^\intercal = (r_j(P_1),r_j(P_2),\dots,r_j(P_n))$\\
		\item \textbf{P} Matrice di Google per righe se:
		\begin{itemize}
			\item $p_{ij}= \frac{1}{P_i}$ se $P_i$ si connette con la pagina $P_j$
			\item $p_{ij}=0$ altrimenti
		\end{itemize}
		\item La precedente iterazione si può riscrivere come: $$ \pi_j^\intercal = \pi_{j-1}^\intercal P$$\\
		\item metodo delle potenze
	\end{itemize}
\end{frame}

%%%%%%%%%%%%%%%%%%%%%%%%%%%%%%%%%%%%%%%%%%%%%%%%%%%%%

\begin{frame}
	\frametitle{Esempio di graph web}
	\begin{columns}
		\begin{column}{0.4\textwidth}
			\includegraphics[width=.9\textwidth]{figure/graph.png}
		\end{column}
		\begin{column}{0.6\textwidth}
			\vspace{1cm}
			$\textbf{P} = \begin{bmatrix}
			0           & \frac{1}{2} & \frac{1}{2}& 0          & 0          & 0           \\[0.3em]
			\frac{1}{2} & 0           & \frac{1}{2}& 0          & 0          & 0           \\[0.3em]
			0           & \frac{1}{2} & 0          & \frac{1}{2}& 0          & 0           \\[0.3em]
			0           & 0           & 0          & 0          & \frac{1}{2}& \frac{1}{2} \\[0.3em]
			0           & 0           & \frac{1}{2}& \frac{1}{2}& 0          & 0           \\[0.3em]
			0           & 0           & 0          & 0          & 0          & 1           \\[0.3em]
			\end{bmatrix}$
		\end{column}
	\end{columns}
\end{frame}

%%%%%%%%%%%%%%%%%%%%%%%%%%%%%%%%%%%%%%%%%%%%%%%%%%%%%

\begin{frame}
	\begin{itemize}
		\item Se il limite esiste, il vettore PageRank è definito $$ \pi^\intercal = \lim_{j\to\infty}\pi_j^\intercal$$
		\item la \emph{i}-sima componente del vettore PageRank è il punteggio(pagerank) della pagina $P_i$
		\item Per assicurare la convergenza del processo iterativo la matrice \emph{P} deve essere modificata
	\end{itemize}
\end{frame}

%%%%%%%%%%%%%%%%%%%%%%%%%%%%%%%%%%%%%%%%%%%%%%%%%%%%%

\begin{frame}
	\begin{itemize}
		\item La matrice di Google per righe \textbf{P} è 
		\begin{itemize}
			\item non-negativa
			\item somma degli elementi sulle righe pari a zero\footnote{nodi \emph{dangling}} o uno
		\end{itemize}
		\item Se la matrice \textbf{P} ha tutte le righe con somma pari a uno allora si parla di matrice stocastica:
		\begin{itemize}
			\item autovalore dominante uguale a 1
			\item iterazione PageRank converge all'autovettore sinistro normalizzato $\pi^\intercal=\pi^\intercal \textbf{P}$ t.c. $\pi^\intercal \mathbb{1} = 1$
		\end{itemize}
	\end{itemize}
\end{frame}

%%%%%%%%%%%%%%%%%%%%%%%%%%%%%%%%%%%%%%%%%%%%%%%%%%%%%

\begin{frame}
	\frametitle{Esempio nodo dangling}
	\begin{columns}
		\begin{column}{0.4\textwidth}
			\includegraphics[width=.8\textwidth]{figure/graphDangling.png}
		\end{column}
		\begin{column}{0.6\textwidth}
			\vspace{1cm}
			$\textbf{P} = \begin{bmatrix}
			0           & \frac{1}{2} & \frac{1}{2}& 0          & 0          & 0           \\[0.3em]
			\frac{1}{2} & 0           & \frac{1}{2}& 0          & 0          & 0           \\[0.3em]
			0           & \frac{1}{2} & 0          & \frac{1}{2}& 0          & 0           \\[0.3em]
			0           & 0           & 0          & 0          & \frac{1}{2}& \frac{1}{2} \\[0.3em]
			0           & 0           & \frac{1}{2}& \frac{1}{2}& 0          & 0           \\[0.3em]
			\textcolor{red}{0}&\textcolor{red}{0}&\textcolor{red}{0}&\textcolor{red}{0}&\textcolor{red}{0}& \textcolor{red}{0}\\[0.3em]
			\end{bmatrix}$
		\end{column}
	\end{columns}
	Il nodo 6 è un nodo dangling in quando non ha outlink
\end{frame}

%%%%%%%%%%%%%%%%%%%%%%%%%%%%%%%%%%%%%%%%%%%%%%%%%%%%%
\begin{frame}
	\frametitle{Trasformazione Matrice di Google per righe}
	\begin{itemize}
		\item Stocastica
		\begin{itemize}
			\item Sostituire ad ogni riga nulla il vettore $\frac{\mathbb{1}^\intercal}{n}$
			\item La nuova matrice stocastica si indica con $\bar{\textbf{P}}$\\~\\
		\end{itemize}
		
		\item Irriducibile
				\begin{itemize}
					\item Aggiungere una matrice di perturbazione $\textbf{E} = \frac{\mathbb{11}^\intercal}{n}$
					\item La nuova matrice sarà uguale a  $$\bar{\bar{\textbf{P}}} = \alpha\bar{\textbf{P}} + (1-\alpha)\textbf{E} ~~~~~~~~ \alpha \in [0,1]$$
				\end{itemize}
	\end{itemize}
\end{frame}

%%%%%%%%%%%%%%%%%%%%%%%%%%%%%%%%%%%%%%%%%%%%%%%%%%%%%
\begin{frame}
	\begin{itemize}
		\item La matrice di Google attualmente utilizzata è ottenuta
		considerando la matrice di perturbazione  $\textbf{E} = \mathbb{1v}^\intercal$ dove $v^\intercal$ è un vettore di personalizzazione dell'utente $$\bar{\bar{\textbf{P}}} = \alpha\bar{\textbf{P}} + (1-\alpha)\mathbb{1v}^\intercal ~~~~~~~~ \alpha \in [0,1]$$
	\end{itemize}
\end{frame}

%%%%%%%%%%%%%%%%%%%%%%%%%%%%%%%%%%%%%%%%%%%%%%%%%%%%%
\section{Implementazione}
\begin{frame}
	
\end{frame}
%%%%%%%%%%%%%%%%%%%%%%%%%%%%%%%%%%%%%%%%%%%%%%%%%%%%%
\begin{frame}
	
\end{frame}
%%%%%%%%%%%%%%%%%%%%%%%%%%%%%%%%%%%%%%%%%%%%%%%%%%%%%
\section{Esempio}
%%%%%%%%%%%%%%%%%%%%%%%%%%%%%%%%%%%%%%%%%%%%%%%%%%%%%
\section{}
\begin{frame}
%	basicstyle=\fontsize{8}{10}\selectfont \ttfamily,%
\begin{center}
Grazie per l'attenzione.
\end{center}
\end{frame}
\end{document}
