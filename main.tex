\documentclass{beamer}
\usepackage{beamerthemesplit}
\usepackage{booktabs}
\usepackage{graphicx}
\usepackage{transparent}
\usepackage[italian]{babel}
\usepackage[utf8x]{inputenc}
\usepackage{listings}
\usepackage{tikz}
\usepackage{amsfonts}
\usepackage{pgfplots}
\usepackage{scalefnt}
\usepackage{color}
\usepackage{xcolor}
\title[PAGERANK]{PageRank}
\institute{
\begin{small}
Corso di Laurea in Informatica Magistrale
\end{small}}
\author{\textbf{Simone Rutigliano}}
\date{\tiny{\today}}

\usebackgroundtemplate{
%    \transparent{0.12}{
     \includegraphics[width=\paperwidth, height=\paperheight]{./figure/escher_hands_tr.png}
%    }
}

%\usetheme{Hannover}
\usetheme{Copenhagen}
\usecolortheme{seahorse}
\usecolortheme{rose}
%\usetheme{Frankfurt}
%\usecolortheme{beetle}

%\useoutertheme[subsection=false]{smoothbars}
%\useoutertheme[subsection=false]{smoothtree}
\useoutertheme{shadow}
\setbeamercovered{dynamic}

\pgfdeclareimage[height=1cm]{logo}{figure/logo}
\logo{\pgfuseimage{logo}}

\begin{document}

%%%%%%%%%%%%%%%%%%%%%%%%%%%%%%%%%%%%%%%%%%%%%%%%%%%%%

\begin{frame}
\maketitle
\end{frame}

%%%%%%%%%%%%%%%%%%%%%%%%%%%%%%%%%%%%%%%%%%%%%%%%%%%%%

\begin{frame}
\frametitle{Outline}
	\tableofcontents
\end{frame}

%%%%%%%%%%%%%%%%%%%%%%%%%%%%%%%%%%%%%%%%%%%%%%%%%%%%%

\section{Introduzione}
\begin{frame}
\frametitle{Definizione}
In 1998 (the same year that HITS was born), Google founders Larry Page and Sergey Brin formulated
their PageRank concept and made it the basis for their search engine [15]. As stated on the Google web
site
By avoiding the inherent weaknesses of HITS, PageRank has been responsible for elevating Google to the
position of the world's preeminent search engine.
\end{frame}

%%%%%%%%%%%%%%%%%%%%%%%%%%%%%%%%%%%%%%%%%%%%%%%%%%%%%

\begin{frame}
	\frametitle{Definizione}
	\begin{itemize}
		\item Formulato nel 1998 da Larry Page e Sergey Brin
		\item Algoritmo di ricerca di Google : " The heart of our software is PageRank \texttrademark\dots it provides the basis for all of our web search tools.''
		\item Supera gli svantaggi di HITS
	\end{itemize}
\end{frame}

%%%%%%%%%%%%%%%%%%%%%%%%%%%%%%%%%%%%%%%%%%%%%%%%%%%%%

\begin{frame}
	\frametitle{Idea di base}
	I voti (link) da siti importanti dovrebbero avere un peso maggiore dei voti (link) da siti meno importanti, e l'importanza di un voto (link) da una qualunque sorgente dovrebbe essere attenuato dal numero dei siti che la sorgente vota
\end{frame}


%%%%%%%%%%%%%%%%%%%%%%%%%%%%%%%%%%%%%%%%%%%%%%%%%%%%%

\begin{frame}
\frametitle{Definition}
\begin{figure}
	\includegraphics[width=.3\textwidth]{figure/graph.png}
\end{figure}
\end{frame}

%%%%%%%%%%%%%%%%%%%%%%%%%%%%%%%%%%%%%%%%%%%%%%%%%%%%%

\begin{frame}
	\frametitle{Formalismo}
	\begin{itemize}
		\item Indicata con $P$ una generica pagina
		\item $B_P$ = \{ insieme delle pagine che puntano a P \}
		\item si definisce il punteggio $r(P)$ di $P$: $$\sum_{Q \in B_P}\frac{r(Q)}{|Q|}$$ dove \\ $B_P=\{$ insieme di tutte le pagine puntanti a $P\}$ \\ $|Q|$ = numero degli outlink di $Q$
	\end{itemize}
\end{frame}

%%%%%%%%%%%%%%%%%%%%%%%%%%%%%%%%%%%%%%%%%%%%%%%%%%%%%

\begin{frame}
	\begin{itemize}
		\item Se abbiamo \emph{n} pagine $P_1,P_2,\dots,P_n$ ed assegniamo a ciascuna pagina un arbitrario punteggio iniziale $r_0(P_i)=\frac{1}{n}$
		\item Il punteggio $r(P)$ può essere calcolato mediante la seguente iterazione: $$r_j(P_i)= \sum_{Q \in B_{P_i}}\frac{r_{j-1}(Q)}{|Q|} ~~~~~~~~ j=1,2,3,\dots$$
	\end{itemize}
\end{frame}


%%%%%%%%%%%%%%%%%%%%%%%%%%%%%%%%%%%%%%%%%%%%%%%%%%%%%

\begin{frame}
	\begin{itemize}
		\item Ponendo: $\pi_j^T = (r_j(P_1),r_j(P_2),\dots,r_j(P_n))$
		\item $P$ Matrice di Google per righe se:
		\begin{itemize}
			\item $p_{ij}= \frac{1}{P_i}$ se $P_i$ si connette con la pagina $P_j$
			\item $p_{ij}=0$ altrimenti
		\end{itemize}
		\item La precedente iterazione si può riscrivere come: $$ \pi_j^T = \pi_{j-1}^T P$$
	\end{itemize}
\end{frame}


%%%%%%%%%%%%%%%%%%%%%%%%%%%%%%%%%%%%%%%%%%%%%%%%%%%%%

\begin{frame}
	\begin{itemize}
		\item Se il limite esiste, il vettore PageRank è definito $$ \pi^T = \limsup\pi_j^T$$
		\item la \emph{i}-sima componente del vettore PageRank è il punteggio(pagerank) della pagina $P_i$
		\item Per assicurare la convergenza del processo iterativo la matrice \emph{P} deve essere modificata
	\end{itemize}
\end{frame}


%%%%%%%%%%%%%%%%%%%%%%%%%%%%%%%%%%%%%%%%%%%%%%%%%%%%%

\begin{frame}
\frametitle{Esempio - Resource Description Framework}
Considerando la frase:\\~\\
\begin{center} \emph{Tarantino is the director of the Django Unchained.} \\~\\
\end{center}
L'affermazione può essere suddivisa come: \\~\\
\begin{tabular}{ l | l }
 Soggetto (Risorsa) & Django Unchained \\
 Predicato (Proprietà) & director \\
 Oggetto (Risorsa) & Tarantino \\
\end{tabular}
\end{frame}

%%%%%%%%%%%%%%%%%%%%%%%%%%%%%%%%%%%%%%%%%%%%%%%%%%%%%
\section{Funzionamento}
%%%%%%%%%%%%%%%%%%%%%%%%%%%%%%%%%%%%%%%%%%%%%%%%%%%%%
\section{Implementazione}
%%%%%%%%%%%%%%%%%%%%%%%%%%%%%%%%%%%%%%%%%%%%%%%%%%%%%
\section{Esempio}
%%%%%%%%%%%%%%%%%%%%%%%%%%%%%%%%%%%%%%%%%%%%%%%%%%%%%
\section{}
\begin{frame}
%	basicstyle=\fontsize{8}{10}\selectfont \ttfamily,%
\begin{center}
Grazie per l'attenzione.
\end{center}
\end{frame}
\end{document}
